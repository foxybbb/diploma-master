The rapid development of mobile robotics is driving growth in various sectors, including industrial automation, service applications, and autonomous vehicles. One of the most important challenges in this field is ensuring the safe movement of robots in dynamic environments populated by people, other robots, and obstacles. Despite the widespread use of traditional sensors such as cameras and LiDAR, performance can deteriorate significantly in adverse environmental conditions, such as poor lighting, dust, fog, rain and snow. This limitation highlights the need for other control technologies capable of providing reliable perception of the environment regardless of environmental conditions.


Millimetre wave (mmWave) radar technology has emerged as a promising alternative. Although historically associated with the automotive and aerospace industries, recent advances have led to the development of compact and inexpensive ‘radar-on-a-chip’ solutions, making millimetre wave radar increasingly accessible for a wide range of applications. 


The aim of this research is to investigate the feasibility and use of a compact millimetre-wave radar for real-time motion planning of mobile robots in dynamic indoor environments. The main part of the work is devoted to the adaptation and development of signal processing methods.
The main objectives of this thesis are:

\begin{enumerate}
    \item To analyze the suitability of mmWave radar for mobile robot motion planning in dynamic settings.
    \item To select and integrate appropriate mmWave radar hardware with a mobile robot platform.
    \item To develop and implement radar signal processing algorithms for object detection, velocity estimation, and direction-of-arrival estimation.
    \item To integrate the radar-derived environmental information into a ROS2-based motion planning system.
    \item To experimentally evaluate the performance of the developed system in detecting static and dynamic obstacles and in enabling adaptive navigation.
\end{enumerate}


The master's thesis consists of the following sections: Chapter 1 provides historical background, problem statement, motivation, and research objectives. Chapter 2 presents an overview of recent advances in millimetre radar technology and its application in robotics. Chapter 3 describes in detail the selection of the millimetre radar and its characteristics, and presents the results of experiments on the detection and characterisation of objects using the radar. Chapter 4 describes the integration of the radar into a mobile robot platform, the kinematic model of the robot, the software architecture for motion planning, and the results of experiments on navigation in dynamic environments. The final chapter 5 summarises the work, discusses limitations, and suggests directions for further research.